\documentclass[DIN, pagenumber=false, fontsize=11pt, parskip=half]{scrartcl}

\usepackage{amsmath}
\usepackage{amsfonts}
\usepackage{amssymb}
\usepackage{enumitem}
\usepackage[utf8]{inputenc}
\usepackage[ngerman]{babel}
\usepackage[T1]{fontenc} 
\usepackage{commath}
\usepackage{xcolor}
\usepackage{booktabs}
\usepackage{float}
\usepackage{tikz-timing}
\usepackage{tikz}
\usepackage{multirow}
\usepackage{colortbl}
\usepackage{xstring}
\usepackage{circuitikz}
\usepackage{listings}
\usepackage[final]{pdfpages}
\usepackage{subcaption}
\usepackage{import}
\usepackage{cleveref}
\usepackage{bm}

\newcommand{\Z}[0]{\mathbb{Z}}
\newcommand{\ZZ}{\mathrm{Z\kern-.3em\raise-0.5ex\hbox{Z}}}
\newcommand{\gq}[1]{\grqq #1\glqq}

%---------------------Stolen from https://github.com/aul12/Kryptologie/blob/master/Uebung01/main.tex
\definecolor{deepblue}{rgb}{0,0,0.5}
\definecolor{deepred}{rgb}{0.6,0,0}
\definecolor{deepgreen}{rgb}{0,0.5,0}

\newcommand\pythonstyle{\lstset{
    language=Python,
    basicstyle=\scriptsize,
    otherkeywords={self},             % Add keywords here
    keywordstyle=\color{deepblue},
    emph={MyClass,__init__},          % Custom highlighting
    emphstyle=\color{deepred},    % Custom highlighting style
    stringstyle=\color{deepgreen},
    frame=tb,                         % Any extra options here
    showstringspaces=false,            % 
}}

% Python environment
\lstnewenvironment{python}[1][]
{
\pythonstyle
\lstset{#1}
}
{}

% Python for external files
\newcommand\pythonexternal[2][]{{
\pythonstyle
\lstinputlisting[#1]{#2}}}

% Python for inline
\newcommand\pythoninline[1]{{\pythonstyle\lstinline!#1!}}

%--------------------------------------

\title{Kryptologie Blatt 1}
\author{Tim Luchterhand}

\begin{document}
    \maketitle
    \setcounter{section}{1}
    \section{Affine Verschlüsselung}
    \begin{enumerate}
        \item Es gibt $\varphi(26) = 12$ verschiedene Möglichkeiten für $k_1$ und 26 verschiedene Möglichkeiten für $k_2$. Insgesamt
              gibt es also $12 \cdot 26 = 312$ verschiedene Schlüssel $k$.
        \item Damit $E$ injektiv ist, muss gelten $\text{ggT}(k_1, 26) \stackrel{!}{=} 1$. Dann $\exists \bar{k_1} : k_1 \cdot 
              \bar{k_1} \equiv 1 \mod 26$. $k_2 \in \{0, \dots, 25\} =: K_2$ darf beliebig gewählt werden. $k_2 \in \Z \setminus K_2$
              wäre auch zulässig, allerdings $\forall k_2 \in \Z \setminus K_2 \ \exists \tilde{k_2} \in K_2 : k_2 \equiv \tilde{k_2} \mod 26$.
        \item Für $D(k, x)$ gilt allgemein:
              \begin{equation*}
                  D(k, x) = \bar{k_1} (x - k_2) \mod 26, \ \text{mit} \ \bar{k_1} \ \text{dem miltiplikativem Inversen zu} \ k_1
              \end{equation*}
              Für $k = (3, 5)$ gilt im speziellen $\bar{k_1} = 9$, denn $9 \cdot 3 = 27 \equiv 1 \mod 26$.
              $D$ ist eindeutig, da
              \begin{align*}
                  D\left(k, E(k, x)\right) &= \bar{k_1} \left(E(k, x) - k_2\right) \mod 26 \\
                  &\equiv \bar{k_1} \left(k_1 x + k_2 - k_2 \right) \mod 26 \\
                  &\equiv \bar{k_1} k_1 x \mod 26 \\
                  &\equiv x \mod 26
              \end{align*}

    \end{enumerate}
    \newpage
    \section{Viginère-Verschlüsselung}
    Diese Aufgabe wurde am Computer mithilfe von Python gelöst. Der Code ist dieser Abgabe als 'decryptVigenere.py' angefügt. Der Quellcode ist
    außerdem am Ende des Dokuments zu sehen. Die Ausgabe des Skripts ist:
    \begin{lstlisting}
Following triples exist multiple times:
[('qgy', 2), ('gyx', 2), ('yxo', 2), ('oqg', 2)]
The distances between these triplets are
[25, 25, 25, 35]
Therefore the most likely key length is 5
Analysis of the symbol frequencies yields following possible keys:
most likely
['l', 'm', 'c', 'h', 't']
second most likely
['m', 'i', 'l', 'u', 'p']
One can easily see that the keyword must be 'licht'. 
The decrypted text is:
manrasieredenkopfdesbotentaetowierediebotschaftinseine
kopfhautwartebisdiehaarenachgewachsensindundschickeihnaufdiereise
    \end{lstlisting}

    Quellcode:
    \pythonexternal{decryptVigenere.py}
\end{document}