\documentclass[DIN, pagenumber=false, fontsize=11pt, parskip=half]{scrartcl}

\usepackage{amsmath}
\usepackage{amsfonts}
\usepackage{amssymb}
\usepackage{enumitem}
\usepackage[utf8]{inputenc}
\usepackage[ngerman]{babel}
\usepackage[T1]{fontenc} 
\usepackage{commath}
\usepackage{xcolor}
\usepackage{booktabs}
\usepackage{float}
\usepackage{tikz-timing}
\usepackage{tikz}
\usepackage{multirow}
\usepackage{colortbl}
\usepackage{xstring}
\usepackage{circuitikz}
\usepackage{listings}
\usepackage[final]{pdfpages}
\usepackage{subcaption}
\usepackage{import}
\usepackage[german]{cleveref}
\usepackage{bm}
\usepackage{moresize}
\usepackage{fancyvrb}
\usepackage{stmaryrd}

\newcommand{\Z}[0]{\mathbb{Z}}
\newcommand{\N}[0]{\mathbb{N}}
\newcommand{\R}[0]{\mathbb{R}}
\newcommand{\ZZ}{\mathrm{Z\kern-.3em\raise-0.5ex\hbox{Z}}}
\newcommand{\gq}[1]{\grqq #1\glqq}
\newcommand{\Ent}{\text{H}}
\newcommand{\MI}{\text{I}}
\newcommand{\ggt}{\text{ggT}}
\newcommand{\kgv}{\text{kgV}}
\newcommand{\rem}{\ \text{Rest} \ }
\newcommand{\congTo}[3][]{\stackrel{#1}{\equiv} #2\mod #3}
\newcommand{\incfig}[2][\columnwidth]{%
    \def\svgwidth{#1}
    \import{./}{#2.eps_tex}
}
\newcommand{\Qed}{\begin{flushright}
    q.e.d.
\end{flushright}}

%---------------------Stolen from https://github.com/aul12/Kryptologie/blob/master/Uebung01/main.tex
\definecolor{deepblue}{rgb}{0,0,0.5}
\definecolor{deepred}{rgb}{0.6,0,0}
\definecolor{deepgreen}{rgb}{0,0.5,0}

\newcommand\pythonstyle{\lstset{
    language=Python,
    basicstyle=\scriptsize,
    otherkeywords={self},             % Add keywords here
    keywordstyle=\color{deepblue},
    emph={MyClass,__init__},          % Custom highlighting
    emphstyle=\color{deepred},    % Custom highlighting style
    stringstyle=\color{deepgreen},
    frame=tb,                         % Any extra options here
    showstringspaces=false,            % 
}}

% Python environment
\lstnewenvironment{python}[1][]
{
\pythonstyle
\lstset{#1}
}
{}

% Python for external files
\newcommand\pythonexternal[2][]{{
\pythonstyle
\lstinputlisting[#1]{#2}}}

% Python for inline
\newcommand\pythoninline[1]{{\pythonstyle\lstinline!#1!}}

%--------------------------------------

\title{Kryptologie Blatt 10}
\author{Tim Luchterhand}

\begin{document}
    \maketitle
    \section{Schwellwertsystem}
    Angenommen wird, dass die Teilschlüssel durch ein Shamir-Schwellwertsystem erzeugt wurden. Um nun an den Schlüssel zu kommen, muss das
    Polynom $P(x) = z_2 x^2 + z_1 x + c$ bestimmt werden, sodass alle gegebenen Punkte eine Lösung des Polynoms sind. Der gesuchte Schlüssel ist
    dann $P(0) = c$. Folgende Gleichungen gelten:
    \begin{align}
        z_2 + z_1 + c \congTo[!]{1}{13} \\
        25 z_2 + 5 z_1 + c \congTo{12 z_2 + 5 z_1 + c}{13} \congTo[!]{5}{13}\\
        144 z_2 + 12 z_1 + c \congTo{z_2 + 12 z_1 + c}{13} \congTo[1]{3}{13}
    \end{align}
    Aus (3) - (1) folgt $11 z_1 \congTo{9}{13}$. Berechne das Inverse zu $11 \mod 13$:
    \begin{align*}
        13 &= 1 \cdot 11 + 2 \\
        11 &= 5 \cdot2 + 1 \\
        1 &= 11 - 5 \cdot (13 - 1 \cdot 11) = -5 \cdot 13 + 6 \cdot 11 \\
        &\Rightarrow 11^{-1} \congTo{6}{13}
    \end{align*}
    Damit folgt, dass $z_1 = 9 \cdot 6 \congTo{10}{23}$. Aus (2) + (3) folgt:
    \begin{align*}
        13 z_2 + 17 z_1 + 2 c \congTo{8}{13} \\
        \Leftrightarrow 4 z_1 + 2 c \congTo{8}{13} \\ 
        \hat{\Leftrightarrow} \ 40 + 2 c \congTo{8}{13} \\
        \Leftarrow 1 + 2 c \congTo{8}{13} \\
        \Leftrightarrow 2 c \congTo{7}{13}
    \end{align*}
    Das Inverse zu $2 \mod 13$ ist 7. $\Rightarrow c = 7 \cdot 7 \congTo{10}{13}$
    Das Geheimnis ist als 10.

    \section{Elliptische Kurve}
    $y^2 \congTo{x^3+x}{23}$
    \begin{enumerate}[label=\alph*)]
        \item Prüfe, ob die Punkte $(9, 18)$ und $(3, 10)$ auf der Kurve liegen durch Nachrechnen:
              \begin{equation*}
                  18^2 = 324 \congTo{2}{23} \ \wedge \ 9^3 + 9 = 738 \congTo{2}{23}
              \end{equation*}
              $\Rightarrow$ der erste Punkt liegt auf der Kurve.
              \begin{equation*}
                  10^2 = 100 \congTo{8}{23} \ \wedge \ 3^3 + 3 = 30 \congTo{7}{23}
              \end{equation*}
              Der zweite Punkt liegt nicht auf der Kurve.
        \item Zunächst wird die Steigung $\lambda$ zwischen den beiden Punkten $P$ und $Q$ berechnet:
              \begin{equation*}
                  \lambda = (y_2 - y_1) \cdot (x_2 - x_1)^{-1} \hat{=} (5 - 5) \cdot (5 - 1)^{-1} = 0
              \end{equation*}
              Die endgültigen Koordinaten $(x_3, y_3)$ ergeben sich dann durch:
              \begin{align*}
                  x_3 &= \lambda^2 - x_1 - x_2 \hat{=} -1 - 9 \congTo{13}{23} \\
                  y_3 &= \lambda \cdot (x_1 - x_3) - y_1 \hat{=} -5 \congTo{18}{23} \\
                  &\Rightarrow P \diamond Q = (13, 18)
              \end{align*}
        \item Hier berechnet sich $\lambda$ durch:
              \begin{equation*}
                  \lambda = (3 \cdot x_1^2 + a) \cdot (2 \cdot y_1) \hat{=} (3 \cdot 1^2 + 1) \cdot (2 \cdot 5) = 4 \cdot 10^{-1}
              \end{equation*}
              Berechne das multiplikative Inverse zu $10 \mod 23$ durch den extggT:
              \begin{align*}
                  23 &= 2 \cdot 10 + 3 \\
                  10 &= 3 \cdot 3 + 1  \\
                  1 &= 10 - 3 \cdot 3 = 10 - 3 \cdot (23 - 2 \cdot 10) \\
                  &= 10 - 3 \cdot 23 + 6 \cdot 10 = -3 \cdot 23 + 7 \cdot 10 \\
                  &\Rightarrow 10^{-1} \congTo{7}{23} \Rightarrow \lambda = 4 \cdot 7 \congTo{5}{23}
              \end{align*}
              $(x_3, y_3)$ berechnen sich dann analog zu b) zu:
              \begin{align*}
                  x_3 &= 5^2 - 1 - 1 = 23 \congTo{0}{23} \\
                  y_3 &= 5 \cdot (1 - 0) - 5 = 0 \\
                  &\Rightarrow P^2 = (0, 0)
              \end{align*}
    \end{enumerate}
\end{document}