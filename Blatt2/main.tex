\documentclass[DIN, pagenumber=false, fontsize=11pt, parskip=half]{scrartcl}

\usepackage{amsmath}
\usepackage{amsfonts}
\usepackage{amssymb}
\usepackage{enumitem}
\usepackage[utf8]{inputenc}
\usepackage[ngerman]{babel}
\usepackage[T1]{fontenc} 
\usepackage{commath}
\usepackage{xcolor}
\usepackage{booktabs}
\usepackage{float}
\usepackage{tikz-timing}
\usepackage{tikz}
\usepackage{multirow}
\usepackage{colortbl}
\usepackage{xstring}
\usepackage{circuitikz}
\usepackage{listings}
\usepackage[final]{pdfpages}
\usepackage{subcaption}
\usepackage{import}
\usepackage{cleveref}
\usepackage{bm}
\usepackage{moresize}

\newcommand{\Z}[0]{\mathbb{Z}}
\newcommand{\ZZ}{\mathrm{Z\kern-.3em\raise-0.5ex\hbox{Z}}}
\newcommand{\gq}[1]{\grqq #1\glqq}
\newcommand{\Ent}{\text{H}}
\newcommand{\incfig}[2][\columnwidth]{%
    \def\svgwidth{#1}
    \import{./}{#2.eps_tex}
}

%---------------------Stolen from https://github.com/aul12/Kryptologie/blob/master/Uebung01/main.tex
\definecolor{deepblue}{rgb}{0,0,0.5}
\definecolor{deepred}{rgb}{0.6,0,0}
\definecolor{deepgreen}{rgb}{0,0.5,0}

\newcommand\pythonstyle{\lstset{
    language=Python,
    basicstyle=\scriptsize,
    otherkeywords={self},             % Add keywords here
    keywordstyle=\color{deepblue},
    emph={MyClass,__init__},          % Custom highlighting
    emphstyle=\color{deepred},    % Custom highlighting style
    stringstyle=\color{deepgreen},
    frame=tb,                         % Any extra options here
    showstringspaces=false,            % 
}}

% Python environment
\lstnewenvironment{python}[1][]
{
\pythonstyle
\lstset{#1}
}
{}

% Python for external files
\newcommand\pythonexternal[2][]{{
\pythonstyle
\lstinputlisting[#1]{#2}}}

% Python for inline
\newcommand\pythoninline[1]{{\pythonstyle\lstinline!#1!}}

%--------------------------------------

\title{Kryptologie Blatt 2}
\author{Tim Luchterhand}

\begin{document}
    \maketitle
    \setcounter{section}{1}
    \section{Entropie}
    \begin{enumerate}
        \item 
            \begin{eqnarray*}
                \ZZ \ \Ent(X | Y) &=& \Ent(X, Y) - \Ent(Y) \\
                \Ent(X | Y) &\stackrel{\text{Def}}{=}& \sum_y{p(y) \Ent(X | Y = y)} \\
                &\stackrel{\text{Def}}{=}& - \sum_y{p(y) \sum_x{p(x | Y = y) \log_2\left(p(x | Y = 0)\right)}} \\
                &=& - \sum_y{\sum_x{p(y) p(x | Y = y) \log_2\left(p(x | Y = 0)\right)}} \\
                &\stackrel{\text{Prod. Regel}}{=}& - \sum_y{\sum_x{p(x, y) \log_2\left(p(x | Y = 0)\right)}} \\
                &\stackrel{\text{Def bed. Wahrsch.}}{=}& - \sum_y{\sum_x{p(x, y) \log_2\left(\frac{p(x,y)}{p(y)}\right)}} \\
                &=& - \sum_y{\sum_x{p(x, y) \left(\log_2\left(p(x, y)\right) - \log_2\left(p(y)\right)\right)}} \\
                &\stackrel{\text{Def } \Ent(X, Y)}{=}& \Ent(X, Y) + \sum_y{\sum_x{p(x, y) \log_2\left(p(y)\right)}} \\
                &=& \Ent(X, Y) + \sum_y{\log_2\left(p(y)\right) \sum_x{p(x, y) }} \\
                &\stackrel{\text{Marginalisierung}}{=}& \Ent(X, Y) + \sum_y{\log_2\left(p(y)\right) p(y)} \\
                &\stackrel{\text{Def } \Ent(Y)}{=}& \Ent(X, Y) - \Ent(Y)
            \end{eqnarray*}
        \item
            \begin{eqnarray*}
                \ZZ \ \Ent(X) &\leq& \Ent(X, Y) \\
                \Ent(Y | X) &\stackrel{1.}{=}& \Ent(Y, X) - \Ent(X) \\
                &=& \Ent(X, Y) - H(X) \\
                &\Leftrightarrow& \Ent(X) = \Ent(X, Y) - \Ent(Y | X) \\
                &\stackrel{\Ent > 0}{\leq}& \Ent(X, Y)
            \end{eqnarray*}
    \end{enumerate}

    \setcounter{section}{4}

    \section{Schieberegister}
    \begin{enumerate}[label=\alph*)]
        \item  $ $
        \begin{figure}[H]
            \centering
            \incfig[\textwidth]{Schieberegister}
            \caption{Schaltplan des Schieberegisters}
        \end{figure}
        \item Die maximale Periodenlänge eines Schieberegisters mit vier Speichereinheiten beträgt $2^4 - 1 = 15$.
        \item Für die Seeds 0000 bzw. 1111 beträgt die Periodenlänge 1. Für alle anderen Werte beträgt sie 7. Diese
              Werte wurden durch Ausprobieren ermittelt. Der dafür verwendete Python-Code lautet:
                \pythonexternal{testReg.py}
    \end{enumerate}
\end{document}