\documentclass[DIN, pagenumber=false, fontsize=11pt, parskip=half]{scrartcl}

\usepackage{amsmath}
\usepackage{amsfonts}
\usepackage{amssymb}
\usepackage{enumitem}
\usepackage[utf8]{inputenc}
\usepackage[ngerman]{babel}
\usepackage[T1]{fontenc} 
\usepackage{commath}
\usepackage{xcolor}
\usepackage{booktabs}
\usepackage{float}
\usepackage{tikz-timing}
\usepackage{tikz}
\usepackage{multirow}
\usepackage{colortbl}
\usepackage{xstring}
\usepackage{circuitikz}
\usepackage{listings}
\usepackage[final]{pdfpages}
\usepackage{subcaption}
\usepackage{import}
\usepackage{cleveref}
\usepackage{bm}
\usepackage{moresize}

\newcommand{\Z}[0]{\mathbb{Z}}
\newcommand{\ZZ}{\mathrm{Z\kern-.3em\raise-0.5ex\hbox{Z}}}
\newcommand{\gq}[1]{\grqq #1\glqq}
\newcommand{\Ent}{\text{H}}
\newcommand{\MI}{\text{I}}
\newcommand{\ggt}{\text{ggT}}
\newcommand{\rem}{\ \text{Rest} \ }
\newcommand{\congTo}[2]{\equiv #1\mod #2}
\newcommand{\incfig}[2][\columnwidth]{%
    \def\svgwidth{#1}
    \import{./}{#2.eps_tex}
}

%---------------------Stolen from https://github.com/aul12/Kryptologie/blob/master/Uebung01/main.tex
\definecolor{deepblue}{rgb}{0,0,0.5}
\definecolor{deepred}{rgb}{0.6,0,0}
\definecolor{deepgreen}{rgb}{0,0.5,0}

\newcommand\pythonstyle{\lstset{
    language=Python,
    basicstyle=\scriptsize,
    otherkeywords={self},             % Add keywords here
    keywordstyle=\color{deepblue},
    emph={MyClass,__init__},          % Custom highlighting
    emphstyle=\color{deepred},    % Custom highlighting style
    stringstyle=\color{deepgreen},
    frame=tb,                         % Any extra options here
    showstringspaces=false,            % 
}}

% Python environment
\lstnewenvironment{python}[1][]
{
\pythonstyle
\lstset{#1}
}
{}

% Python for external files
\newcommand\pythonexternal[2][]{{
\pythonstyle
\lstinputlisting[#1]{#2}}}

% Python for inline
\newcommand\pythoninline[1]{{\pythonstyle\lstinline!#1!}}

%--------------------------------------

\title{Kryptologie Blatt 3}
\author{Tim Luchterhand}

\begin{document}
    \maketitle
    \setcounter{section}{2}
    \section{Entropie und Transinformation}
    \begin{enumerate}[label=\alph*)]
        \item 
        \begin{align*}
            \Ent(X, Y) ~\stackrel{\text{Def}}{=}& -\sum_{x, y}{p(x, y) \log_2\left(p(x, y)\right)} \\
            \hat{=}& ~-4 \cdot \frac{1}{4} \log_2\left(\frac{1}{4}\right) = 2 \\
            \Ent(X) \stackrel{\text{Def}}{=}& -\sum_x{p(x) \log_2\left(p(x)\right)} \ \text{mit} \ p(x_1) = \frac{3}{4} \ \text{und}
            \ p(x_2) = \frac{1}{4} \\
            \hat{=}& -\frac{3}{4} \log_2\left(\frac{3}{4}\right) - \frac{1}{4} \cdot (-2) \approx 0.81 \\
            \Ent(Y) \stackrel{\text{Def}}{=}& -\sum_y{p(y) \log_2\left(p(y)\right)} \ \text{mit} \ p(y_1) = \frac{1}{2} \ \text{und}
            \ p(x_2) = p(y_3) = \frac{1}{4} \\
            \hat{=}& - \frac{1}{2} \cdot (-1) - 2 \cdot \frac{1}{4} \cdot (-2) = 1 \ \frac{1}{2}
        \end{align*}
        \item 
        \begin{align*}
            \Ent(X|Y) \stackrel{\text{Blatt 2}}{=}& \Ent(X, Y) - \Ent(Y) \hat{=} \ 2 - 1 \ \frac{1}{2} = \frac{1}{2} \\
            \Ent(Y|X) \stackrel{\text{Blatt 2}}{=}& \Ent(X, Y) - \Ent(X) \approx 2 - 0.81 = 1.19\\
        \end{align*}
        \item 
        \begin{equation*}
            \MI(X, Y) = \Ent(Y) - \Ent(Y|X) \approx 1.5 - 1.19 = 0.31
        \end{equation*}
    \end{enumerate}
    \section{Pohlig-Hellman}
    \begin{enumerate}[label=\alph*)]
        \item 
        \begin{align*}
            22 =& ~2 \cdot 9 \rem 4 \\
            9 =& ~2 \cdot 4 \rem 1 \\
            4 =& ~4 \cdot 1 \rem 0 \\
            \Rightarrow \ggt(9, n-1) =& ~\ggt(9, 22) = 1
        \end{align*}
        \item Nutze die vorletzte Zeile von a):
        \begin{align*}
            &9 = 2 \cdot 4 + 1 \Leftrightarrow 1 = 9 - 2 \cdot 4 \\
            \Leftrightarrow& ~1 = 9 - 2 \cdot (22 - 2 \cdot 9) \\
            \Leftrightarrow& ~1 = 5 \cdot 9 - 2 \cdot 22 \\
            \Rightarrow& ~x = -2, ~y = 5 \\
            \Rightarrow& ~d = y = 5 \wedge d \cdot e = 5 \cdot 9 \congTo{1}{22}
        \end{align*}
        \item 
        \begin{align*}
            c = m^e \mod n ~\hat{=}~ 16^9 \mod 23 = 8 \\
            m = c^d \mod n ~\hat{=}~ 8^5 \mod 23 = 16
        \end{align*}
        \item Man erhält folgenden Chiffretext: \newline
        1, 6, 18, 13, 11, 16, 15, 9, 2, 20, 19, 4, 3, 21, 14, 8, 7, 12, 10, 5, 17, 22
        \item Wir der Chiffretext $c$ nach dem Schema $c=\vec{m}^e \mod 23$ berechnet mit 
        $\vec{m} = (1, \dots, 22)$ und $e \in \Z_{22}^*$, dann existieren $\varphi(22) = |\Z_{22}^*| = 10$
        verschiedene Permutationen.
    \end{enumerate}
\end{document}