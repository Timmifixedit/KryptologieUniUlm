\documentclass[DIN, pagenumber=false, fontsize=11pt, parskip=half]{scrartcl}

\usepackage{amsmath}
\usepackage{amsfonts}
\usepackage{amssymb}
\usepackage{enumitem}
\usepackage[utf8]{inputenc}
\usepackage[ngerman]{babel}
\usepackage[T1]{fontenc} 
\usepackage{commath}
\usepackage{xcolor}
\usepackage{booktabs}
\usepackage{float}
\usepackage{tikz-timing}
\usepackage{tikz}
\usepackage{multirow}
\usepackage{colortbl}
\usepackage{xstring}
\usepackage{circuitikz}
\usepackage{listings}
\usepackage[final]{pdfpages}
\usepackage{subcaption}
\usepackage{import}
\usepackage[german]{cleveref}
\usepackage{bm}
\usepackage{moresize}

\newcommand{\Z}[0]{\mathbb{Z}}
\newcommand{\N}[0]{\mathbb{N}}
\newcommand{\R}[0]{\mathbb{R}}
\newcommand{\ZZ}{\mathrm{Z\kern-.3em\raise-0.5ex\hbox{Z}}}
\newcommand{\gq}[1]{\grqq #1\glqq}
\newcommand{\Ent}{\text{H}}
\newcommand{\MI}{\text{I}}
\newcommand{\ggt}{\text{ggT}}
\newcommand{\kgv}{\text{kgV}}
\newcommand{\rem}{\ \text{Rest} \ }
\newcommand{\congTo}[2]{\equiv #1\mod #2}
\newcommand{\incfig}[2][\columnwidth]{%
    \def\svgwidth{#1}
    \import{./}{#2.eps_tex}
}

%---------------------Stolen from https://github.com/aul12/Kryptologie/blob/master/Uebung01/main.tex
\definecolor{deepblue}{rgb}{0,0,0.5}
\definecolor{deepred}{rgb}{0.6,0,0}
\definecolor{deepgreen}{rgb}{0,0.5,0}

\newcommand\pythonstyle{\lstset{
    language=Python,
    basicstyle=\scriptsize,
    otherkeywords={self},             % Add keywords here
    keywordstyle=\color{deepblue},
    emph={MyClass,__init__},          % Custom highlighting
    emphstyle=\color{deepred},    % Custom highlighting style
    stringstyle=\color{deepgreen},
    frame=tb,                         % Any extra options here
    showstringspaces=false,            % 
}}

% Python environment
\lstnewenvironment{python}[1][]
{
\pythonstyle
\lstset{#1}
}
{}

% Python for external files
\newcommand\pythonexternal[2][]{{
\pythonstyle
\lstinputlisting[#1]{#2}}}

% Python for inline
\newcommand\pythoninline[1]{{\pythonstyle\lstinline!#1!}}

%--------------------------------------

\title{Kryptologie Blatt 4}
\author{Tim Luchterhand}

\begin{document}
    \maketitle
    \setcounter{section}{1}
    \section{Gruppenaxiome}
    \paragraph{Neutrales Element}
    Wie aus der Tabelle zu entnehmen ist, ist $\alpha$ das neutrale Element von $G$, denn $\forall g \in G : g \circ \alpha = g$
    \paragraph{Inverses Element}
    $\forall g \in G \ \exists g' \in G : g \circ g' = \alpha$. Wie aus der Tabelle zu entnehmen, gilt $g' = g \ \forall g \in G$. 
    \paragraph{Kommutativität}
    Die Tabelle ist symmetrisch zur Hauptachse. Daraus folgt die Kommutativität.
    \paragraph{Assoziativität}
    Beweis per Brute-Force:
    Um Assoziativität nachzuweisen muss lediglich getestet werden, ob $a \circ (b \circ c) = (a \circ b) \circ c \ \forall a, b, c \in G$ gilt.
    Dies wurde durch folgendes Python-Skript automatisiert. Dabei wurden die Symbole $\alpha, \beta, \dots$ jeweils mit 0, 1, \dots ersetzt, um
    die Abbildungsvorschrift für $\circ$ als Matrix zu realisieren.
    \pythonexternal{bruteforce.py}

    Ausgabe des Skripts:
    \begin{lstlisting}
        a: 0, b: 0, c: 0
        True
        a: 0, b: 0, c: 1
        True
        a: 0, b: 0, c: 2
        True
        a: 0, b: 0, c: 3
        True
        a: 0, b: 1, c: 0
        True
        a: 0, b: 1, c: 1
        True
        a: 0, b: 1, c: 2
        True
        a: 0, b: 1, c: 3
        True
        a: 0, b: 2, c: 0
        True
        a: 0, b: 2, c: 1
        True
        a: 0, b: 2, c: 2
        True
        a: 0, b: 2, c: 3
        True
        a: 0, b: 3, c: 0
        True
        a: 0, b: 3, c: 1
        True
        a: 0, b: 3, c: 2
        True
        a: 0, b: 3, c: 3
        True
        a: 1, b: 0, c: 0
        True
        a: 1, b: 0, c: 1
        True
        a: 1, b: 0, c: 2
        True
        a: 1, b: 0, c: 3
        True
        a: 1, b: 1, c: 0
        True
        a: 1, b: 1, c: 1
        True
        a: 1, b: 1, c: 2
        True
        a: 1, b: 1, c: 3
        True
        a: 1, b: 2, c: 0
        True
        a: 1, b: 2, c: 1
        True
        a: 1, b: 2, c: 2
        True
        a: 1, b: 2, c: 3
        True
        a: 1, b: 3, c: 0
        True
        a: 1, b: 3, c: 1
        True
        a: 1, b: 3, c: 2
        True
        a: 1, b: 3, c: 3
        True
        a: 2, b: 0, c: 0
        True
        a: 2, b: 0, c: 1
        True
        a: 2, b: 0, c: 2
        True
        a: 2, b: 0, c: 3
        True
        a: 2, b: 1, c: 0
        True
        a: 2, b: 1, c: 1
        True
        a: 2, b: 1, c: 2
        True
        a: 2, b: 1, c: 3
        True
        a: 2, b: 2, c: 0
        True
        a: 2, b: 2, c: 1
        True
        a: 2, b: 2, c: 2
        True
        a: 2, b: 2, c: 3
        True
        a: 2, b: 3, c: 0
        True
        a: 2, b: 3, c: 1
        True
        a: 2, b: 3, c: 2
        True
        a: 2, b: 3, c: 3
        True
        a: 3, b: 0, c: 0
        True
        a: 3, b: 0, c: 1
        True
        a: 3, b: 0, c: 2
        True
        a: 3, b: 0, c: 3
        True
        a: 3, b: 1, c: 0
        True
        a: 3, b: 1, c: 1
        True
        a: 3, b: 1, c: 2
        True
        a: 3, b: 1, c: 3
        True
        a: 3, b: 2, c: 0
        True
        a: 3, b: 2, c: 1
        True
        a: 3, b: 2, c: 2
        True
        a: 3, b: 2, c: 3
        True
        a: 3, b: 3, c: 0
        True
        a: 3, b: 3, c: 1
        True
        a: 3, b: 3, c: 2
        True
        a: 3, b: 3, c: 3
        True
    \end{lstlisting}

    \section{$\varphi$-Funktion}
    \begin{enumerate}[label=\alph*)]
        \item $\ZZ \ \varphi(n) = \varphi(r) \varphi(s)$ für $n = r \cdot s, \ r, s \in \Z$ teilerfremd.
        Seien
        \begin{equation*}
            r = \prod_{i=1}^k{r_i^{e_{r,i}}} \ , \ s = \prod_{i=1}^l{s_i^{e_{s,i}}}
        \end{equation*}
        die Primfaktorzerlegungen von $r$ und $s$. Dann gilt:
        \begin{equation*}
            n = r \cdot s = \prod_{i=1}^k{r_i^{e_{r,i}}} \cdot \prod_{i=1}^l{s_i^{e_{s,i}}} = \prod_{i=1}^{k+l}{p_i^{e_{n, i}}}
        \end{equation*}
        mit 
        \begin{align}
            (p_i)_{i=1}^{k+l} &= (r_1, \dots, r_k, s_1, \dots, s_l) \label{eq:p}\\
            \text{und} \ (e_{n,i})_{i=1}^{k+l} &= (e_{r, 1}, \dots, e_{r, k}, e_{s, 1}, \dots, e_{s, l})
        \end{align}
        Durch die Teilerfremdheit von $r$ und $s$ gilt hierbei $p_i \neq p_j$ für $i \neq j$.
        Aus der Vorlesung ist bekannt, dass für $n \in \N$ mit den Primfaktoren $p_1, \dots, p_m$ gilt:
        \begin{equation}
            \varphi(n) = n \cdot \prod_{i=1}^m{\left(1 - \frac{1}{p_i}\right)}
            \label{eq:phi}
        \end{equation}
        Dann folgt für $n = r \cdot s$
        \begin{align*}
            \varphi(n) &= n \cdot \prod_{i = 1}^{k+l}{\left(1 - \frac{1}{p_i}\right)} =
            n \cdot \prod_{i=1}^k{\left(1 - \frac{1}{p_i}\right)} \prod_{i=k+1}^l{\left(1 - \frac{1}{p_i}\right)} \\
            &\stackrel{\ref{eq:p}}{=} n \cdot \prod_{i=1}^k{\left(1 - \frac{1}{r_i}\right)} \prod_{i=1}^l{\left(1 - \frac{1}{s_i}\right)} \\
            & \stackrel{n=r \cdot s}{=} r \cdot \prod_{i=1}^k{\left(1 - \frac{1}{r_i}\right)} \cdot s \cdot \prod_{i=1}^l{\left(1 - \frac{1}{s_i}\right)} \\
            &\stackrel{\ref{eq:phi}}{=} \varphi(r) \cdot \varphi(s)
        \end{align*}
        \begin{flushright}
            q.e.d.
        \end{flushright}

        \newpage

        \item $\ZZ \ \varphi(n) \varphi(n) = \varphi\left(\ggt(n, m)\right) \varphi\left(\kgv(m, n)\right)$ für $n, m \in \N$. \newline
        Fall 1: $\ggt(n, m) = 1$ \newline
        Dann ist $\kgv(n, m) = m \cdot n$ und es folgt:
        \begin{align*}
            \varphi(n) \varphi(m) &\stackrel{\text{3 a)}}{=} \varphi(n \cdot m) = 1 \cdot \varphi\left(\kgv(n, m)\right) \\
            &= \varphi(1) \varphi\left(\kgv(n, m)\right) = \varphi(\ggt(n, m)) \varphi\left(\kgv(n, m)\right)
        \end{align*}
        Anmerkung: $\varphi(1)$ ist per Definition 1.


        Fall 2: $\ggt(n, m) \neq 1$ \newline
        Seien $(p_i)_{i=1}^k$ bzw. $(q_i)_{i=1}^l$ die unterschiedlichen Primfaktoren von $n$ bzw. $m$, also $p_i \neq p_j \ \wedge \ q_i \neq q_j$ für 
        $i \neq j$. Zu jedem Primfaktor $p_i$ bzw. $q_i$ gehöre die Potenz $e_i$ bzw. $f_i$, sodass gilt:
        \begin{equation}
            n = \prod_{i=1}^k{p_i^{e_i}} \ \text{bzw.} \ m = \prod_{i=1}^l{q_i^{f_i}} \label{eq:pfz}
        \end{equation}
        Beide Zahlen haben nach Voraussetzung einen oder mehrere gemeinsame Teiler. oBdA gelte Folgendes für die Primfaktoren $p_i$ und $q_i$:
        \begin{align}
            (p_i)_{i=1}^k &= (p_1, \dots, p_k) \label{eq:nPrime}\\
            (q_i)_{i=1}^l &= (p_1, \dots, p_j, r_1, \dots, r_h) \ \text{mit} \ j \leq k \ \text{und} \ l = j+h \label{eq:mPrime}
        \end{align}
        D.h. $n, m$ besitzen die gemeinsamen Primfaktoren $p_1, \dots, p_j$. $n$ besitzt darüber hinaus die Primfaktoren $ p_{j+1}, \dots, p_k$ und $m$
        dir Primfaktoren $r_1, \dots, r_h$.

        \paragraph{Einschub: Überlegungen zum ggT und kgV} \label{par:Einschub} $ $ \newline
        Bezeichne $P(z)$ im Folgenden die paarweise verschiedenen Primfaktoren von $z$ für $z \in \N$. Der $\ggt(n,m) =: \ggt$ ist die größtmögliche Zahl, 
        die sowohl $n$ als auch $m$ teilt. Aus $\ggt \ | \ n$ folgt, dass $P(\ggt) \subseteq P(n)$. Ebenso muss gelten $P(\ggt) \subseteq P(m)$ damit 
        $\ggt \ | \ m$ sichergestellt ist. Damit folgt im konkreten Fall:
        \begin{equation}
            P(\ggt) = P(n) \cup P(m) \stackrel{\ref{eq:mPrime}}{=} \{p_1, \dots, p_j\} \label{eq:PggT}
        \end{equation}
        Da der ggT die größtmögliche Zahl ist, die $n$ und $m$ teilt, lässt sie sich über ihre Primfaktoren $P(\ggt)$ folgendermaßen darstellen:
        \begin{equation}
            \ggt = \prod_{i=1}^j{p_i^{g_i}} \ \text{mit} \ g_i = \min\{e_i, f_i\} \ \forall i \leq j \label{eq:ggt}
        \end{equation}
        Besitzt man also die vollständige Primfaktorzerlegungen von $n$ und $m$, so ist der ggT das Produkt aus allen gemeinsamen Primfaktoren von $n$ und $m$
        wobei die einzelnen Primfaktoren $p_i$ jeweils die kleinere der zugehörigen Potenzen $e_i$ und $f_i$ besitzen muss. Anders ausgedrückt sind die Faktoren
        $p_i^{g_i}$ alle größten (Primpotenz-) Teiler, die sowohl $n$ als auch $m$ teilen. \newpage

        Für das $\kgv(n, m) =: \kgv$ gilt $n \ | \ \kgv$ und $m \ | \ \kgv$. Daraus folgt, dass $P(\kgv) \supseteq P(n)$ und $P(\kgv) \supseteq m$ gelten muss.
        Für den konkreten Fall bedeutet dies:
        \begin{equation}
            P(\kgv) = P(n) \cup P(m) \stackrel{\ref{eq:nPrime}, \ref{eq:mPrime}}{=} \{p_1, \dots, p_k, r_1, \dots, r_h\} \label{eq:PkgV}
        \end{equation}
        Analog zum ggT lässt sich das kgV ebenfalls aus den Primfaktoren der beiden Zahlen $n$ und $m$ berechnen. Es ist das Produkt aus allen Primfaktoren
        von $n$ und $m$, wobei für die Primfaktoren $p_i, i \leq k$ anstatt $g_i = \min\{e_i, f_i\}$ als Potenz das Maximum von $e_i$ und $f_i$ gewählt wird:
        \begin{equation}
            \kgv = \prod_{i=1}^j{p_i^{h_i}} \prod_{i=j+1}^k{p_i^{e_i}} \prod_{i=1}^h{r_i^{f_i}} \ 
            \text{mit} \ h_i = \max\{e_i, f_i\} \ \forall i \leq j \label{eq:kgv}
        \end{equation}
        Aus \cref{eq:ggt} und \ref{eq:kgv} lässt sich durch Umsortieren folgern
        \begin{align}
            \ggt\cdot \kgv &= \prod_{i=1}^j{p_i^{g_i}} \prod_{i=1}^j{p_i^{h_i}} \prod_{i=j+1}^k{p_i^{e_i}} \prod_{i=1}^h{r_i^{f_i}} \\
            &= \prod_{i=1}^j{p_i^{e_i}} \prod_{i=j+1}^k{p_i^{e_i}} \prod_{i=1}^j{p_i^{f_i}} \prod_{i=1}^h{r_i^{f_i}} \\
            &\stackrel{\ref{eq:mPrime}}{=} \prod_{i=1}^k{p_i^{e_i}} \prod_{i=1}^l{q_i^{f_i}} = n \cdot m \label{eq:ggtTimeskgv}
        \end{align}

        \paragraph{Zurück zum Beweis}
        \begin{align*}
            \varphi(n) \varphi(m) &\stackrel{\ref{eq:phi}}{=} n \cdot \prod_{i=1}^k{\left(1 - \frac{1}{p_i}\right)} 
            \cdot m \cdot \prod_{i=1}^l{\left(1 - \frac{1}{q_i}\right)} \\
            &\stackrel{\ref{eq:mPrime}}{=} n \cdot \prod_{i=1}^k{\left(1 - \frac{1}{p_i}\right)} 
            \cdot m \cdot \prod_{i=1}^j{\left(1 - \frac{1}{p_i}\right)} \prod_{i=1}^h{\left(1 - \frac{1}{r_i}\right)} \\
            &= n \cdot \prod_{i=1}^{j}{\left(1 - \frac{1}{p_i}\right)} \prod_{i=j+1}^k{\left(1 - \frac{1}{p_i}\right)}
            \cdot m \cdot \prod_{i=1}^j{\left(1 - \frac{1}{p_i}\right)} \prod_{i=1}^h{\left(1 - \frac{1}{r_i}\right)} \\
            &= n \cdot \prod_{i=1}^{j}{\left(1 - \frac{1}{p_i}\right)} \cdot m \cdot \prod_{i=1}^j{\left(1 - \frac{1}{p_i}\right)} 
            \prod_{i=1}^h{\left(1 - \frac{1}{r_i}\right)} \prod_{i=j+1}^k{\left(1 - \frac{1}{p_i}\right)}\\
            &\stackrel{\ref{eq:PggT}, \ref{eq:PkgV}}{=} n \cdot \prod_{p \in P(\ggt)}{\left(1 - \frac{1}{p}\right)}
            \cdot m \cdot \prod_{p \in P(\kgv)}{\left(1 - \frac{1}{p}\right)} \\
            &\stackrel{\ref{eq:ggtTimeskgv}}{=} \ggt \cdot \prod_{p \in P(\ggt)}{\left(1 - \frac{1}{p}\right)} 
            \cdot \kgv \prod_{p \in P(\kgv)}{\left(1 - \frac{1}{p}\right)} \\
            &\stackrel{\ref{eq:phi}}{=} \varphi(\ggt) \varphi(\kgv) \\
        \end{align*}
        \begin{flushright}
            q.e.d.
        \end{flushright}
    \end{enumerate}
\end{document}