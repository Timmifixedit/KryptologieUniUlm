\documentclass[DIN, pagenumber=false, fontsize=11pt, parskip=half]{scrartcl}

\usepackage{amsmath}
\usepackage{amsfonts}
\usepackage{amssymb}
\usepackage{enumitem}
\usepackage[utf8]{inputenc}
\usepackage[ngerman]{babel}
\usepackage[T1]{fontenc} 
\usepackage{commath}
\usepackage{xcolor}
\usepackage{booktabs}
\usepackage{float}
\usepackage{tikz-timing}
\usepackage{tikz}
\usepackage{multirow}
\usepackage{colortbl}
\usepackage{xstring}
\usepackage{circuitikz}
\usepackage{listings}
\usepackage[final]{pdfpages}
\usepackage{subcaption}
\usepackage{import}
\usepackage[german]{cleveref}
\usepackage{bm}
\usepackage{moresize}

\newcommand{\Z}[0]{\mathbb{Z}}
\newcommand{\N}[0]{\mathbb{N}}
\newcommand{\R}[0]{\mathbb{R}}
\newcommand{\ZZ}{\mathrm{Z\kern-.3em\raise-0.5ex\hbox{Z}}}
\newcommand{\gq}[1]{\grqq #1\glqq}
\newcommand{\Ent}{\text{H}}
\newcommand{\MI}{\text{I}}
\newcommand{\ggt}{\text{ggT}}
\newcommand{\kgv}{\text{kgV}}
\newcommand{\rem}{\ \text{Rest} \ }
\newcommand{\congTo}[2]{\equiv #1\mod #2}
\newcommand{\incfig}[2][\columnwidth]{%
    \def\svgwidth{#1}
    \import{./}{#2.eps_tex}
}

%---------------------Stolen from https://github.com/aul12/Kryptologie/blob/master/Uebung01/main.tex
\definecolor{deepblue}{rgb}{0,0,0.5}
\definecolor{deepred}{rgb}{0.6,0,0}
\definecolor{deepgreen}{rgb}{0,0.5,0}

\newcommand\pythonstyle{\lstset{
    language=Python,
    basicstyle=\scriptsize,
    otherkeywords={self},             % Add keywords here
    keywordstyle=\color{deepblue},
    emph={MyClass,__init__},          % Custom highlighting
    emphstyle=\color{deepred},    % Custom highlighting style
    stringstyle=\color{deepgreen},
    frame=tb,                         % Any extra options here
    showstringspaces=false,            % 
}}

% Python environment
\lstnewenvironment{python}[1][]
{
\pythonstyle
\lstset{#1}
}
{}

% Python for external files
\newcommand\pythonexternal[2][]{{
\pythonstyle
\lstinputlisting[#1]{#2}}}

% Python for inline
\newcommand\pythoninline[1]{{\pythonstyle\lstinline!#1!}}

%--------------------------------------

\title{Kryptologie Blatt 5}
\author{Tim Luchterhand}

\begin{document}
    \maketitle
    \section{Quadratwurzel}
    Die Quadratwurzeln von $1 \mod 55$ lassen sich über den chinesischen Restsatz berechnen. Da gilt, dass $55 = 5 \cdot 11$ berechnet man jeweils die Quadratwurzeln
    $\sqrt{1} \mod 5$ und $\sqrt{1} \mod 11$. Die Quadratwurzel von 1 bezüglich dieser beiden Module ist jeweils 1 und -1. Die vier Quadratwurzeln von $1 \mod 55$
    berechnen sich dann aus der Rücktransformation aller Permutationen von 1 und -1. Diese wurden durch folgendes Python-Skript berechnet:
    \pythonexternal{transform.py}
    Die Ausgabe des Skripts ist:
    \begin{lstlisting}
Square root of 1 mod 55 is 1
1^2 = 1 mod 55
Square root of 1 mod 55 is 34
34^2 = 1 mod 55
Square root of 1 mod 55 is 21
21^2 = 1 mod 55
Square root of 1 mod 55 is 54
54^2 = 1 mod 55
    \end{lstlisting}

\section{Oberhuber No-Key-Protokoll}
Die Sicherheit des ursprünglichen No-Key-Protokolls von Shamir beruht auf der Tatsache, dass sich der diskrete Logarithmus nicht effizient berechnen lässt. Da bei dieser
Variante keine modulare Exponentiation zum Einsatz kommt, ist das Protokoll angreifbar:
\begin{enumerate}
    \item Zunächst erhält der Angreifer $c = m \cdot a \mod n$. Ohne Kenntnis von $m$ oder $a$ lässt sich hier nicht viel machen. Jedoch lässt sich mithilfe des extggT
          das Inverse $c'$ zu $c$ effizient berechnen.
    \item Als nächstes wird $d = c \cdot b \mod n$ abgefangen. Hier kann nun mit dem Inversen von $c$ multipliziert werden:
          \begin{equation*}
              d \cdot c' \congTo{c \cdot b \cdot c'}{n} \stackrel{\Z^*_n \text{ist abelsche Gruppe}}{\equiv} c \cdot c' \cdot b \mod n \congTo{1 \cdot b}{n}
          \end{equation*}
          Nun besitzt der Angreifer also die geheime Zahl $b$ und kann wiederum das Inverse $b'$ durch den extggT berechnen.
    \item Sobald der Angreifer $e = d \cdot a' \congTo{m \cdot b}{n}$ erhält, kann er die Nachricht $m$ entschlüsseln, indem er mit $b'$ multipliziert:
          \begin{equation*}
              e \cdot b' \congTo{m \cdot b \cdot b'}{n} \congTo{m \cdot 1}{n}
          \end{equation*}
    \item Zusätzlich könnte der Angreifer auch das Inverse zu $m$ berechnen und dadurch and die geheime Zahl $a$ gelangen:
          \begin{equation*}
              c \cdot m' \congTo{m \cdot a \cdot m'}{n} \congTo{a \cdot 1}{n}
          \end{equation*}
          Dadurch kann er nun die Kommunikation in beide Richtungen entschlüsseln.
\end{enumerate}
\end{document}