\documentclass[DIN, pagenumber=false, fontsize=11pt, parskip=half]{scrartcl}

\usepackage{amsmath}
\usepackage{amsfonts}
\usepackage{amssymb}
\usepackage{enumitem}
\usepackage[utf8]{inputenc}
\usepackage[ngerman]{babel}
\usepackage[T1]{fontenc} 
\usepackage{commath}
\usepackage{xcolor}
\usepackage{booktabs}
\usepackage{float}
\usepackage{tikz-timing}
\usepackage{tikz}
\usepackage{multirow}
\usepackage{colortbl}
\usepackage{xstring}
\usepackage{circuitikz}
\usepackage{listings}
\usepackage[final]{pdfpages}
\usepackage{subcaption}
\usepackage{import}
\usepackage[german]{cleveref}
\usepackage{bm}
\usepackage{moresize}
\usepackage{fancyvrb}

\newcommand{\Z}[0]{\mathbb{Z}}
\newcommand{\N}[0]{\mathbb{N}}
\newcommand{\R}[0]{\mathbb{R}}
\newcommand{\ZZ}{\mathrm{Z\kern-.3em\raise-0.5ex\hbox{Z}}}
\newcommand{\gq}[1]{\grqq #1\glqq}
\newcommand{\Ent}{\text{H}}
\newcommand{\MI}{\text{I}}
\newcommand{\ggt}{\text{ggT}}
\newcommand{\kgv}{\text{kgV}}
\newcommand{\rem}{\ \text{Rest} \ }
\newcommand{\congTo}[2]{\equiv #1\mod #2}
\newcommand{\incfig}[2][\columnwidth]{%
    \def\svgwidth{#1}
    \import{./}{#2.eps_tex}
}

%---------------------Stolen from https://github.com/aul12/Kryptologie/blob/master/Uebung01/main.tex
\definecolor{deepblue}{rgb}{0,0,0.5}
\definecolor{deepred}{rgb}{0.6,0,0}
\definecolor{deepgreen}{rgb}{0,0.5,0}

\newcommand\pythonstyle{\lstset{
    language=Python,
    basicstyle=\scriptsize,
    otherkeywords={self},             % Add keywords here
    keywordstyle=\color{deepblue},
    emph={MyClass,__init__},          % Custom highlighting
    emphstyle=\color{deepred},    % Custom highlighting style
    stringstyle=\color{deepgreen},
    frame=tb,                         % Any extra options here
    showstringspaces=false,            % 
}}

% Python environment
\lstnewenvironment{python}[1][]
{
\pythonstyle
\lstset{#1}
}
{}

% Python for external files
\newcommand\pythonexternal[2][]{{
\pythonstyle
\lstinputlisting[#1]{#2}}}

% Python for inline
\newcommand\pythoninline[1]{{\pythonstyle\lstinline!#1!}}

%--------------------------------------

\title{Kryptologie Blatt 6}
\author{Tim Luchterhand}

\begin{document}
    \maketitle
    \setcounter{section}{2}
    \section{Miller-Rabin-Primzahltest}
    Der Primzahltest wurde durch folgendes Python-Skript automatisiert.
    \pythonexternal{primeTest.py}
    Die Ausgabe des Skripts ist
    \VerbatimInput{output.txt}

    \section{Untergruppe}
    Die Aussage gilt im Allgemeinen \textit{nicht}. \\
    Beweis: \\
    Sei $n \neq 2$ eine Primzahl (für den Fall 2 gelten die Voraussetzungen nicht), dann gilt: $n = 1 \cdot n$ und $\ggt(1, n) = 1$, wobei 1 und 
    $n$ offensichtlich ungerade sind. Weiterhin gilt für $n$:
    \begin{equation}
        n = \prod_{i=1}^l{a_i} + 1
        \label{eq:n}
    \end{equation}
    Hierbei sind $a_i, i = 1, \dots, l$ die (nicht notwendigerweise verschiedenen) Primfaktoren von $n - 1$. Um nun $n - 1$ als $k \cdot 2^u$
    darzustellen mit $k \in \Z$ ungerade und $u \geq 1$ muss gelten, dass $u = 1$, denn:
    \begin{equation}
        n - 1 \stackrel{\ref{eq:n}}{=} \prod_{i=1}^l{a_i} = 2 \cdot \prod_{i=2}^l{a_i} = 2^1 \cdot k \ \text{mit} \ k = \prod_{i=2}^l{a_i}
        \label{eq:k}
    \end{equation}
    $k$ ist also offensichtlich ungerade. Da $u$ nur den Wert 1 annehmen kann, gilt $i \in \{0, \dots, u - 1\} \Rightarrow i = 0$. 
    Daraus folgt wiederum, dass $a^{k \cdot 2^i} \congTo{\pm 1}{n} \ \forall a \in \{1, \dots, n - 1\}$, denn:
    \begin{align*}
        a^{k \cdot 2^i} &= a^{k \cdot 2^0} = a^{k} \stackrel{\ref{eq:k}}{=} a^{\frac{n - 1}{2}} = \sqrt{a^{n - 1}} \\
        &\stackrel{n \ \text{prim}}{=} \sqrt{a^{\varphi(n)}} \stackrel{\text{Fermat}}{\equiv} \sqrt{1} \mod n \stackrel{n \ \text{prim}}{\equiv} \pm 1 \mod n
    \end{align*}
    Insgesamt folgt also, dass $|B_n| = |\Z_n^*|$ weshalb $B_n$ keine \textit{echte} Untergruppe sein kann.
\end{document}