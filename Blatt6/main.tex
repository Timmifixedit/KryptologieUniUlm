\documentclass[DIN, pagenumber=false, fontsize=11pt, parskip=half]{scrartcl}

\usepackage{amsmath}
\usepackage{amsfonts}
\usepackage{amssymb}
\usepackage{enumitem}
\usepackage[utf8]{inputenc}
\usepackage[ngerman]{babel}
\usepackage[T1]{fontenc} 
\usepackage{commath}
\usepackage{xcolor}
\usepackage{booktabs}
\usepackage{float}
\usepackage{tikz-timing}
\usepackage{tikz}
\usepackage{multirow}
\usepackage{colortbl}
\usepackage{xstring}
\usepackage{circuitikz}
\usepackage{listings}
\usepackage[final]{pdfpages}
\usepackage{subcaption}
\usepackage{import}
\usepackage[german]{cleveref}
\usepackage{bm}
\usepackage{moresize}
\usepackage{fancyvrb}

\newcommand{\Z}[0]{\mathbb{Z}}
\newcommand{\N}[0]{\mathbb{N}}
\newcommand{\R}[0]{\mathbb{R}}
\newcommand{\ZZ}{\mathrm{Z\kern-.3em\raise-0.5ex\hbox{Z}}}
\newcommand{\gq}[1]{\grqq #1\glqq}
\newcommand{\Ent}{\text{H}}
\newcommand{\MI}{\text{I}}
\newcommand{\ggt}{\text{ggT}}
\newcommand{\kgv}{\text{kgV}}
\newcommand{\rem}{\ \text{Rest} \ }
\newcommand{\congTo}[2]{\equiv #1\mod #2}
\newcommand{\incfig}[2][\columnwidth]{%
    \def\svgwidth{#1}
    \import{./}{#2.eps_tex}
}

%---------------------Stolen from https://github.com/aul12/Kryptologie/blob/master/Uebung01/main.tex
\definecolor{deepblue}{rgb}{0,0,0.5}
\definecolor{deepred}{rgb}{0.6,0,0}
\definecolor{deepgreen}{rgb}{0,0.5,0}

\newcommand\pythonstyle{\lstset{
    language=Python,
    basicstyle=\scriptsize,
    otherkeywords={self},             % Add keywords here
    keywordstyle=\color{deepblue},
    emph={MyClass,__init__},          % Custom highlighting
    emphstyle=\color{deepred},    % Custom highlighting style
    stringstyle=\color{deepgreen},
    frame=tb,                         % Any extra options here
    showstringspaces=false,            % 
}}

% Python environment
\lstnewenvironment{python}[1][]
{
\pythonstyle
\lstset{#1}
}
{}

% Python for external files
\newcommand\pythonexternal[2][]{{
\pythonstyle
\lstinputlisting[#1]{#2}}}

% Python for inline
\newcommand\pythoninline[1]{{\pythonstyle\lstinline!#1!}}

%--------------------------------------

\title{Kryptologie Blatt 6}
\author{Tim Luchterhand}

\begin{document}
    \maketitle
    \setcounter{section}{2}
    \section{Miller-Rabin-Primzahltest}
    Der Primzahltest wurde durch folgendes Python-Skript automatisiert.
    \pythonexternal{primeTest.py}
    Die Ausgabe des Skripts ist
    \VerbatimInput{output.txt}

    \newpage

    \section{Untergruppe}
    Im Folgenden seien $i, u$ und $k$ wie in der Aufgabenstellung definiert.
    Um zu zeigen, dass $B_n$ eine echte Untergruppe von $\Z_n^*$ ist, muss folgendes gelten:
    \begin{enumerate}
        \item $B_n \subset \Z_n^*$
        \item $a \cdot b \mod n \in B_n \ \forall a, b \in B_n$
        \item $\forall a \in B_n \ \exists a' \in B_n : a \cdot a' \congTo{1}{n}$
    \end{enumerate}
    Zunächst soll 2. gezeigt werden: \\
    Seien $a, b \in B_n$ und es gelte $a \cdot b \congTo{z}{n}$. Dann gilt:
    \begin{align*}
        z^{k \cdot 2^i} &= (a \cdot b)^{k \cdot 2^i} = a^{k \cdot 2^i} \cdot b^{k \cdot 2^i} \\
        &\congTo{(\pm 1) \cdot (\pm 1)}{n} = \pm 1 \\
        &\Rightarrow z \in B_n
    \end{align*}
    Als nächstes wird die Existenz des inversen Elements gezeigt. Sei $a \in B_n$ und es gelte $a \cdot a' \congTo{1}{n}$. Offensichtlich gilt $1 \in B_n$.
    Daraus folgt:
    \begin{align*}
        1^{k \cdot 2^i} &= (a \cdot a')^{k \cdot 2^i} = a^{k \cdot 2^i} \cdot a'^{k \cdot 2^i} \\
        &\congTo{(\pm 1) \cdot a'^{k \cdot 2^i}}{n} \stackrel{!}{\equiv} 1 \mod n \\
        &\Rightarrow a'^{k \cdot 2^i} \stackrel{!}{\equiv} \mp 1 \mod n \Rightarrow a' \in B_n
    \end{align*}
    Für 1. muss gezeigt werden, dass $a \in B_n \Rightarrow a \in \Z_n^*$ Also muss gezeigt werden: $a^{k \cdot 2^i} \congTo{\pm 1}{n} \Rightarrow \ggt(a, n) = 1$.
    Ohne näheres Wissen über $u, i$ und die Elemente in $B_n$ konnte ich das leider nicht zeigen. Ein Ansatz war, die Implikation in umgekehrte Richtung zu zeigen,
    also $\ggt(a, n) = d > 1 \Rightarrow a^{k \cdot 2^i} \not\congTo{\pm 1}{n}$:
    \begin{equation*}
        a^{k \cdot 2^i} \not\congTo{\pm 1}{n} \Rightarrow a^{k \cdot 2^i} = l \cdot n \pm 1 \Leftrightarrow l = \frac{a ^{k \cdot 2^i} \pm 1}{n} \notin \Z
    \end{equation*}
    Man müsste also zeigen, dass $l$ keine ganze Zahl sein kann. Hierbei wird aber kein Wissen über $i$ und $u$ mit eingebracht. Und ich konnte leider nicht
    nachvollziehen, wie $i$ und $u$ zusammenhängen.

    Möglicherweise lässt sich auch zeigen, dass $|B_n|$ ein Teiler von $|\Z_n^*|$ ist. Dann würde nach Lagrange folgen, dass $B_n$ eine Teilmenge von $\Z_n^*$
    ist. Allerdings konnte ich hier keine konkreten Aussagen über die Mächtigkeiten der beiden Gruppen treffen. Für die Mächtigkeit von $\Z_n^*$ gilt lediglich
    $|\Z_n^*| = \varphi(n) = \varphi(p) \cdot \varphi(q)$, da gilt, dass $n = p \cdot q$ und $\ggt(p, q) = 1$. Über $|B_n|$ lässt sich möglicherweise über die
    Ordnung des Elements $a$ argumentieren, da stets gilt, dass die Ordnung eine Teiler der Mächtigkeit ist. Auch hier fehlte mir der Zusammenhang von $u$ und $i$.
\end{document}