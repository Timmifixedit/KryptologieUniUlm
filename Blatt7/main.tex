\documentclass[DIN, pagenumber=false, fontsize=11pt, parskip=half]{scrartcl}

\usepackage{amsmath}
\usepackage{amsfonts}
\usepackage{amssymb}
\usepackage{enumitem}
\usepackage[utf8]{inputenc}
\usepackage[ngerman]{babel}
\usepackage[T1]{fontenc} 
\usepackage{commath}
\usepackage{xcolor}
\usepackage{booktabs}
\usepackage{float}
\usepackage{tikz-timing}
\usepackage{tikz}
\usepackage{multirow}
\usepackage{colortbl}
\usepackage{xstring}
\usepackage{circuitikz}
\usepackage{listings}
\usepackage[final]{pdfpages}
\usepackage{subcaption}
\usepackage{import}
\usepackage[german]{cleveref}
\usepackage{bm}
\usepackage{moresize}
\usepackage{fancyvrb}

\newcommand{\Z}[0]{\mathbb{Z}}
\newcommand{\N}[0]{\mathbb{N}}
\newcommand{\R}[0]{\mathbb{R}}
\newcommand{\ZZ}{\mathrm{Z\kern-.3em\raise-0.5ex\hbox{Z}}}
\newcommand{\gq}[1]{\grqq #1\glqq}
\newcommand{\Ent}{\text{H}}
\newcommand{\MI}{\text{I}}
\newcommand{\ggt}{\text{ggT}}
\newcommand{\kgv}{\text{kgV}}
\newcommand{\rem}{\ \text{Rest} \ }
\newcommand{\congTo}[3][]{\stackrel{#1}{\equiv} #2\mod #3}
\newcommand{\incfig}[2][\columnwidth]{%
    \def\svgwidth{#1}
    \import{./}{#2.eps_tex}
}
\newcommand{\Qed}{\begin{flushright}
    q.e.d.
\end{flushright}}

%---------------------Stolen from https://github.com/aul12/Kryptologie/blob/master/Uebung01/main.tex
\definecolor{deepblue}{rgb}{0,0,0.5}
\definecolor{deepred}{rgb}{0.6,0,0}
\definecolor{deepgreen}{rgb}{0,0.5,0}

\newcommand\pythonstyle{\lstset{
    language=Python,
    basicstyle=\scriptsize,
    otherkeywords={self},             % Add keywords here
    keywordstyle=\color{deepblue},
    emph={MyClass,__init__},          % Custom highlighting
    emphstyle=\color{deepred},    % Custom highlighting style
    stringstyle=\color{deepgreen},
    frame=tb,                         % Any extra options here
    showstringspaces=false,            % 
}}

% Python environment
\lstnewenvironment{python}[1][]
{
\pythonstyle
\lstset{#1}
}
{}

% Python for external files
\newcommand\pythonexternal[2][]{{
\pythonstyle
\lstinputlisting[#1]{#2}}}

% Python for inline
\newcommand\pythoninline[1]{{\pythonstyle\lstinline!#1!}}

%--------------------------------------

\title{Kryptologie Blatt 7}
\author{Tim Luchterhand}

\begin{document}
    \maketitle
    \setcounter{section}{1}
    \section{Modulare Exponentiation und $\mathbf{\bm{\varphi}(n)}$}
    Im Folgenden gelte: $n = p \cdot q$ mit $p, q > 2$ prim, $p \neq q$. Es gelte oBdA $x \in \Z_n^*$. Wenn nicht, dann existiert wegen $\ggt(x, n) = 1$
    ein $\tilde{x} : x \congTo{\tilde{x}}{n}$. Die Rechnung mit $\tilde{x}$ ist äquivalent zur Rechnung mit $x$, denn x lässt sich schreiben als 
    $k \cdot n + \tilde{x}$. Dann gilt für die modulare Multiplikation:
    \begin{equation}
        a \cdot x \mod n = a \cdot (k \cdot n + \tilde{x}) \mod n = a k n + a \tilde{x} \mod n = a \cdot \tilde{x} \mod n 
        \label{eq:obda}
    \end{equation}
    Diese Äquivalenz gilt ebenfalls für die modulare Exponentiation als wiederholte Anwendung der modularen Multiplikation.
    \begin{enumerate}[label=(\roman*)]
        \item $\ZZ \ x^{\varphi(n)/2} \congTo{1}{p}$ und $x^{\varphi(n)/2} \congTo{1}{q}$. \\
              OBdA sei $x < p$ (Siehe allgemeine Begründung oben, \cref{eq:obda}). Es gilt also $x \in \Z_p^*$, da $p$ prim. Dann gilt für $a, k \in \Z$
              auch folgendes:
              \begin{equation}
                  x^{\varphi(q) \cdot k} \stackrel{p \ \text{prim}}{=} \left(x^{q - 1}\right)^k \congTo[\text{Fermat}]{1^k}{p} = 1
                  \label{eq:fermatP}
              \end{equation}
              Da $q$ ungerade ist, ist $q - 1$ gerade und somit $\frac{q - 1}{2} \in \Z$. Deshalb gilt:
              \begin{equation}
                  x^{\varphi(p) \cdot \frac{q - 1}{2}} = x^{(p - 1) \cdot (q - 1) / 2} = x^{\varphi(n) / 2} \congTo[\ref{eq:fermatP}]{1}{p}
              \end{equation}
              $x^{\varphi(n)/2} \congTo{1}{q}$ lässt sich analog zeigen. \Qed
        \item $\ZZ \ x^{\varphi(n)/2} \congTo{1}{n}$ \\
              Es gilt:
              \begin{align}
                  x^{\varphi(n)} &= \left(x^{\varphi(n)/2}\right)^2 \congTo[\text{Fermat}]{1^2}{n} \\
                  &\Rightarrow x^{\varphi(n)/2} \congTo{\pm 1}{n}
              \end{align}
              Angenommen $x^{\varphi(n)/2} \congTo{-1}{n}$. Sei $y \in \Z : y \congTo{-1}{n}$. Daraus folgt für $k \in \Z$:
              \begin{align}
                  y &= k \cdot n - 1 = k p q - 1 = p \cdot (k \cdot q) - 1 = p \cdot \tilde{k} -1 \\
                  &\Rightarrow y \congTo{-1}{p} \label{eq:mOneInP}
              \end{align}
              Dann folgt wiederum $x^{\varphi(n)/2} \congTo{- 1}{n} \Rightarrow x^{\varphi(n)/2} \congTo{- 1}{p}$. Dies widerspricht allerdings (i), 
              weshalb dann zwingendermaßen $x^{\varphi(n)/2} \congTo{1}{n}$ gelten muss. \Qed
    \end{enumerate}
\end{document} 