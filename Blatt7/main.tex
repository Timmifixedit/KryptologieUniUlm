\documentclass[DIN, pagenumber=false, fontsize=11pt, parskip=half]{scrartcl}

\usepackage{amsmath}
\usepackage{amsfonts}
\usepackage{amssymb}
\usepackage{enumitem}
\usepackage[utf8]{inputenc}
\usepackage[ngerman]{babel}
\usepackage[T1]{fontenc} 
\usepackage{commath}
\usepackage{xcolor}
\usepackage{booktabs}
\usepackage{float}
\usepackage{tikz-timing}
\usepackage{tikz}
\usepackage{multirow}
\usepackage{colortbl}
\usepackage{xstring}
\usepackage{circuitikz}
\usepackage{listings}
\usepackage[final]{pdfpages}
\usepackage{subcaption}
\usepackage{import}
\usepackage[german]{cleveref}
\usepackage{bm}
\usepackage{moresize}
\usepackage{fancyvrb}
\usepackage{stmaryrd}

\newcommand{\Z}[0]{\mathbb{Z}}
\newcommand{\N}[0]{\mathbb{N}}
\newcommand{\R}[0]{\mathbb{R}}
\newcommand{\ZZ}{\mathrm{Z\kern-.3em\raise-0.5ex\hbox{Z}}}
\newcommand{\gq}[1]{\grqq #1\glqq}
\newcommand{\Ent}{\text{H}}
\newcommand{\MI}{\text{I}}
\newcommand{\ggt}{\text{ggT}}
\newcommand{\kgv}{\text{kgV}}
\newcommand{\rem}{\ \text{Rest} \ }
\newcommand{\congTo}[3][]{\stackrel{#1}{\equiv} #2\mod #3}
\newcommand{\incfig}[2][\columnwidth]{%
    \def\svgwidth{#1}
    \import{./}{#2.eps_tex}
}
\newcommand{\Qed}{\begin{flushright}
    q.e.d.
\end{flushright}}

%---------------------Stolen from https://github.com/aul12/Kryptologie/blob/master/Uebung01/main.tex
\definecolor{deepblue}{rgb}{0,0,0.5}
\definecolor{deepred}{rgb}{0.6,0,0}
\definecolor{deepgreen}{rgb}{0,0.5,0}

\newcommand\pythonstyle{\lstset{
    language=Python,
    basicstyle=\scriptsize,
    otherkeywords={self},             % Add keywords here
    keywordstyle=\color{deepblue},
    emph={MyClass,__init__},          % Custom highlighting
    emphstyle=\color{deepred},    % Custom highlighting style
    stringstyle=\color{deepgreen},
    frame=tb,                         % Any extra options here
    showstringspaces=false,            % 
}}

% Python environment
\lstnewenvironment{python}[1][]
{
\pythonstyle
\lstset{#1}
}
{}

% Python for external files
\newcommand\pythonexternal[2][]{{
\pythonstyle
\lstinputlisting[#1]{#2}}}

% Python for inline
\newcommand\pythoninline[1]{{\pythonstyle\lstinline!#1!}}

%--------------------------------------

\title{Kryptologie Blatt 7}
\author{Tim Luchterhand}

\begin{document}
    \maketitle
    \setcounter{section}{1}
    \section{Modulare Exponentiation und $\mathbf{\bm{\varphi}(n)}$}
    Im Folgenden gelte: $n = p \cdot q$ mit $p, q > 2$ prim, $p \neq q$. Es gelte oBdA $x \in \Z_n^*$. Wenn nicht, dann existiert wegen $\ggt(x, n) = 1$
    ein $\tilde{x} : x \congTo{\tilde{x}}{n}$. Die Rechnung mit $\tilde{x}$ ist äquivalent zur Rechnung mit $x$, denn x lässt sich schreiben als 
    $k \cdot n + \tilde{x}$. Dann gilt für die modulare Multiplikation:
    \begin{equation}
        a \cdot x \mod n = a \cdot (k \cdot n + \tilde{x}) \mod n = a k n + a \tilde{x} \mod n = a \cdot \tilde{x} \mod n 
        \label{eq:obda}
    \end{equation}
    Diese Äquivalenz gilt ebenfalls für die modulare Exponentiation als wiederholte Anwendung der modularen Multiplikation.
    \begin{enumerate}[label=(\roman*)]
        \item $\ZZ \ x^{\varphi(n)/2} \congTo{1}{p}$ und $x^{\varphi(n)/2} \congTo{1}{q}$. \\
              OBdA sei $x < p$ (Siehe allgemeine Begründung oben, \cref{eq:obda}). Es gilt also $x \in \Z_p^*$, da $p$ prim. Dann gilt für $k \in \Z$
              auch folgendes:
              \begin{equation}
                  x^{\varphi(p) \cdot k} = \left(x^{\varphi(p)}\right)^k \congTo[\text{Fermat}]{1^k}{p} \congTo{1}{p}
                  \label{eq:fermatP}
              \end{equation}
              Da $q$ ungerade ist, ist $q - 1$ gerade und somit $\frac{q - 1}{2} \in \Z$. Deshalb gilt:
              \begin{equation}
                  x^{\varphi(p) \cdot \frac{q - 1}{2}} = x^{\varphi(p) \cdot k} \congTo[\ref{eq:fermatP}]{1}{p}, \ \text{mit} \ k = \frac{q - 1}{2} \in \Z
              \end{equation}
              $x^{\varphi(n)/2} \congTo{1}{q}$ lässt sich analog zeigen. \Qed
        \item $\ZZ \ x^{\varphi(n)/2} \congTo{1}{n}$ \\
              Es gilt:
              \begin{align}
                  x^{\varphi(n)} &= \left(x^{\varphi(n)/2}\right)^2 \congTo[\text{Fermat}]{1^2}{n} \\
                  \Rightarrow x^{\varphi(n)/2} &\congTo{\pm 1}{n}
              \end{align}
              Angenommen $x^{\varphi(n)/2} \congTo{-1}{n}$. Sei $y \in \Z : y \congTo{-1}{n}$. Daraus folgt für $k \in \Z$:
              \begin{align}
                  y &= k \cdot n - 1 = k p q - 1 = p \cdot (k \cdot q) - 1 = p \cdot \tilde{k} -1 \\
                  \Rightarrow y &\congTo{-1}{p} \label{eq:mOneInP}
              \end{align}
              Dann folgt wiederum $x^{\varphi(n)/2} \congTo{- 1}{n} \Rightarrow x^{\varphi(n)/2} \congTo{- 1}{p}$. Dies widerspricht allerdings (i), 
              weshalb dann zwingendermaßen $x^{\varphi(n)/2} \congTo{1}{n}$ gelten muss. \Qed
        \item $\ZZ \ e \cdot d \congTo{1}{\frac{\varphi(n)}{2}} \Rightarrow x^{e \cdot d} \congTo{x}{n}$. \\
              Offensichtlich gilt: $e \cdot d = \frac{k \cdot \varphi(n)}{2} + 1, \ k \in \Z$. Daraus folgt:
              \begin{align}
                  x^{e \cdot d} &= x^{\frac{k \cdot \varphi(n)}{2} + 1} = x \cdot x^{\frac{k \cdot \varphi(n)}{2}} \\
                   &= x \cdot \left(x^{\varphi(n)/2}\right)^k \congTo[\text{ii}]{x \cdot 1^k}{n} \congTo{x}{n}
              \end{align}
              \Qed
        \item TODO
    \end{enumerate}
    \section{Modulare Exponentiation die Zweite}
    Sein  im Folgenden $p \neq q$ zwei Primzahlen, $n = p \cdot q$. Es gilt also:
    \begin{equation}
        \varphi(n) \stackrel{\text{p, q \ \text{teilerfremd}}}{=} \varphi(p) \cdot \varphi(q) \stackrel{p, q \ \text{prim}}{=} (p - 1) \cdot (q - 1)
        \label{eq:phi}
    \end{equation}
    \begin{enumerate}[label=(\roman*)]
        \item $\ZZ$ für $a, k, m \in \Z :  m = k \cdot \varphi(n), \ \ggt(a, n) = 1$ gilt: $a^m \congTo{1}{p}$ \\ und $a^m \congTo{1}{q}$ \\
              \begin{align}
                  a^m &= a^{k \cdot \varphi(n)} = \left(a^{\varphi(n)}\right)^k \stackrel{\ref{eq:phi}}{=} \left(a^{\varphi(p)}\right)^{k \cdot \varphi(q)} \\
                  &= \left(a^{\varphi(p)}\right)^{\tilde{k}} \congTo[\text{Fermat}]{1^{\tilde{k}}}{p} \congTo{1}{p}
              \end{align}
              $a^m \congTo{1}{q}$ folgt analog. \Qed
        \item Es gelte $\ggt(a, n) \geq 1, \ m = k \cdot \varphi(n)$ mit $k \in \Z$ \\
              $\ZZ \ a^{m + 1} \congTo{a}{p} \wedge a^{m + 1} \congTo{1}{q}$ \\
              Sei oBdA $\ggt(a, n) = p$. Dann folgt $a \congTo{0}{p} \wedge \ggt(a, q) = 1$. Es gilt:
              \begin{align}
                  \label{eq:ggtP}
                  a^{m + 1} &= a \cdot a^m = a \cdot l \congTo{0}{p} \ \text{mit} \ l = a^m \in \Z \Rightarrow a^{m + 1} \congTo{a}{p} \\
                  a^{m + 1} &= a \cdot a^m = a \cdot a^{k \cdot \varphi(n)} \stackrel{\ref{eq:phi}}{=} a \cdot \left(a^{\varphi(q)}\right)^{k \cdot \varphi(p)}
                  \congTo{a \cdot 1^{k \cdot \varphi(p)}}{q} \congTo{a}{q}
              \end{align}
              Der Fall $\ggt(a, n) = q$ folgt analog. Falls gilt, dass $a = l \cdot n, \ l \in \Z$, dann folgt $a \congTo{0}{q} \wedge a \congTo{0}{q}$ und
              die Behauptung folgt analog zu \cref{eq:ggtP}. Für $\ggt(a, n) = 1$ folgt die Behauptung aus (i), indem beide Seiten mit $a$ multipliziert werden. \Qed
        \item Sei $(e, d)$ das RSA-Tupel zum Modul $n$. \\ $\ZZ \ a^{ed} \congTo{a}{n} \ \forall a \in \Z$ \\
              Für $(e, d)$ gilt: $e \cdot d \congTo{1}{\varphi(n)} \Rightarrow e \cdot d = k \cdot \varphi(n) + 1, \ k \in \Z$. \\
              \textbf{Fall $\mathbf{\ggt(a, n) = 1}$:}
              \begin{equation}
                  a^{ed} = a^{k \cdot \varphi(n) + 1} = a \cdot \left(a^{\varphi(n)}\right)^k \congTo[\ggt(a, n) = 1]{a \cdot 1^k}{n} \congTo{a}{n}
              \end{equation}
              \textbf{Fall $\mathbf{\ggt(a, n) \neq 1}$:} \\
              OBdA sei $\ggt(a, n) = p$ und deshalb $a = \tilde{a} \cdot p$. Dann gilt für $m := k \cdot \varphi(n)$:
              \begin{align}
                  a^{ed} &= a^{m + 1} = a \cdot a^m \congTo{a}{n} \\
                  &\Leftrightarrow \tilde{a} \cdot p \cdot a^m \congTo{\tilde{a} \cdot p}{n} \\
                  &\stackrel{\text{Kürzungsregel}}{\Leftrightarrow} \tilde{a} \cdot a^m \congTo{\tilde{a}}{q} \\
                  &\stackrel{\ggt(\tilde{a}, q) = 1}{\Leftrightarrow} a^m \congTo{1}{q} \\
                  &\Leftrightarrow a^{m + 1} \congTo{a}{q} \label{eq:taut}
              \end{align}
              Die Behauptung führt also auf \cref{eq:taut} und das ist nach (i) eine Tautologie. Für den Fall $a = l \cdot n, \ l \in \Z$ folgt
              $a \congTo{0}{n}$ und deshalb gilt:
              \begin{equation}
                  a^{ed} = a \cdot a^{k \cdot \varphi(n)} = a \cdot s \congTo{0}{n}, \ s = a^{k \cdot \varphi(n)} \in \Z \Rightarrow a^{ed} \congTo{a}{n}
              \end{equation}
              \Qed
    \end{enumerate}
\end{document} 