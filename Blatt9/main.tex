\documentclass[DIN, pagenumber=false, fontsize=11pt, parskip=half]{scrartcl}

\usepackage{amsmath}
\usepackage{amsfonts}
\usepackage{amssymb}
\usepackage{enumitem}
\usepackage[utf8]{inputenc}
\usepackage[ngerman]{babel}
\usepackage[T1]{fontenc} 
\usepackage{commath}
\usepackage{xcolor}
\usepackage{booktabs}
\usepackage{float}
\usepackage{tikz-timing}
\usepackage{tikz}
\usepackage{multirow}
\usepackage{colortbl}
\usepackage{xstring}
\usepackage{circuitikz}
\usepackage{listings}
\usepackage[final]{pdfpages}
\usepackage{subcaption}
\usepackage{import}
\usepackage[german]{cleveref}
\usepackage{bm}
\usepackage{moresize}
\usepackage{fancyvrb}
\usepackage{stmaryrd}

\newcommand{\Z}[0]{\mathbb{Z}}
\newcommand{\N}[0]{\mathbb{N}}
\newcommand{\R}[0]{\mathbb{R}}
\newcommand{\ZZ}{\mathrm{Z\kern-.3em\raise-0.5ex\hbox{Z}}}
\newcommand{\gq}[1]{\grqq #1\glqq}
\newcommand{\Ent}{\text{H}}
\newcommand{\MI}{\text{I}}
\newcommand{\ggt}{\text{ggT}}
\newcommand{\kgv}{\text{kgV}}
\newcommand{\rem}{\ \text{Rest} \ }
\newcommand{\congTo}[3][]{\stackrel{#1}{\equiv} #2\mod #3}
\newcommand{\incfig}[2][\columnwidth]{%
    \def\svgwidth{#1}
    \import{./}{#2.eps_tex}
}
\newcommand{\Qed}{\begin{flushright}
    q.e.d.
\end{flushright}}

%---------------------Stolen from https://github.com/aul12/Kryptologie/blob/master/Uebung01/main.tex
\definecolor{deepblue}{rgb}{0,0,0.5}
\definecolor{deepred}{rgb}{0.6,0,0}
\definecolor{deepgreen}{rgb}{0,0.5,0}

\newcommand\pythonstyle{\lstset{
    language=Python,
    basicstyle=\scriptsize,
    otherkeywords={self},             % Add keywords here
    keywordstyle=\color{deepblue},
    emph={MyClass,__init__},          % Custom highlighting
    emphstyle=\color{deepred},    % Custom highlighting style
    stringstyle=\color{deepgreen},
    frame=tb,                         % Any extra options here
    showstringspaces=false,            % 
}}

% Python environment
\lstnewenvironment{python}[1][]
{
\pythonstyle
\lstset{#1}
}
{}

% Python for external files
\newcommand\pythonexternal[2][]{{
\pythonstyle
\lstinputlisting[#1]{#2}}}

% Python for inline
\newcommand\pythoninline[1]{{\pythonstyle\lstinline!#1!}}

%--------------------------------------

\title{Kryptologie Blatt 9}
\author{Tim Luchterhand}

\begin{document}
    \maketitle
    \section{Graphisomorphie}
    \begin{enumerate}[label=\alph*)]
        \item $ $
              \begin{figure}[H]
                \centering
                  \begin{subfigure}[l]{0.49\textwidth}
                    \centering
                      \incfig[\textwidth]{G1}
                      \caption{Graph $G_1$}
                  \end{subfigure}  
                  \begin{subfigure}[l]{0.49\textwidth}
                    \centering
                      \incfig[\textwidth]{H}
                      \caption{Graph $H$}
                  \end{subfigure}  
              \end{figure}
        \item Im Falle von $b = 1$ sendet P die Permutation $\sigma$, die bereits auf dem Blatt angegeben ist.
    \end{enumerate}

    \section{Arbeitsbeschaffungsmaßnahme für Oberhuber}
    \begin{enumerate}[label=\alph*)]
        \item In diesem Fall müsste Peter $z = y$ gesendet haben, wobei für $y$ gelten muss: $y^2 \congTo{\tilde{y}}{n}$ Nachrechnen ergibt:
              \begin{equation*}
                  y^2 = z^2 \hat{=} \ 63^2 \congTo{42}{77} = \tilde{y}
              \end{equation*}
              Peter hat also die korrekte Antwort gesendet.
        \item Hier müsste Peter $z = x \circ y$ gesendet haben. Dies entspricht konkret $z = x \cdot y \mod n$. Oberhuber muss nun prüfen, ob
              $\tilde{x} \cdot \tilde{y} \congTo[?]{z^2}{n}$. Nachrechnen:
              \begin{equation*}
                  \tilde{x} \cdot \tilde{y} \hat{=} \ 37 \cdot 15 \congTo{16}{77} \not\congTo{64}{n}
              \end{equation*}
              Peter ist also kein legitimer Prover. Es fällt außerdem aber auf, dass $z^2 \congTo{15}{n}$. $z$ ist also eine Quadratwurzel von
              $\tilde{y}$. Peter hat also als nicht legitimer Prover gehofft, dass Oberhuber nur die Challenge $b = 0$ erwartet, auf die er
              wie auch schon beim ersten mal die Antwort parat gehabt hätte.
    \end{enumerate}
\end{document}